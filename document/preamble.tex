\sloppy % настройка переноса слов


% Добавляем гипертекстовое оглавление в PDF и метаданные
\usepackage[
bookmarks=true, colorlinks=true, unicode=true,
urlcolor=black,linkcolor=black, anchorcolor=black,
citecolor=black, menucolor=black, filecolor=black,
pdfstartpage=1,
    pdftex,
            pdfauthor={\CYRZ\cyro\cyrl\cyro\cyrt\cyro\cyrv\CYRV\CYRV\ \CYRM\CYRS7},
            pdftitle={\CYRD\CYRI\CYRP\CYRL\CYRO\CYRM\ \the\year },
            pdfsubject={},
            pdfkeywords={},            
]{hyperref}
%% возможные значения по умолнанию
% Приложение:           pdfproducer={pdfTeX-1.40.25},
% Производитель PDF:    pdfcreator={LaTeX with hyperref}
%%
\urlstyle{same} % Текст ссылки имеет обычный формат шрифта

% Изменение начертания шрифта --- после чего выглядит таймсоподобно.
% apt-get install scalable-cyrfonts-tex

\IfFileExists{cyrtimes.sty}
    {
        \usepackage{cyrtimespatched}
    }
    {
        % А если Times нету, то будет CM...
    }


\usepackage{graphicx}   % Пакет для включения рисунков
\DeclareGraphicsExtensions{.jpg,.pdf,.png}
% \graphicspath{{Source/img/}}    % расположение рисунков % было перенесено в main
\usepackage{float} % Для опции фигур [H]

\usepackage{subfigure} %https://latex-tutorial.com/subfigure-latex/
%%
%Необходима для правильности отображения ссылок на на subfigure при нумерации figure по главам
% https://tex.stackexchange.com/questions/115712/figure-numbering-with-section-number-problem-with-subfigure
\makeatletter 
\renewcommand\p@subfigure{\thefigure}
\makeatother
%%

% Установка мм как меры длины для псевдорисунков
\unitlength=1mm 


% Произвольная нумерация списков.
\usepackage{enumerate}


% списки которые не зависят от linespread
%определить среду для компактного перечисления
\newenvironment{cenumerate}{
\setlength{\parskip}{0pt}
\begin{enumerate}
\setlength{\itemsep}{0pt}
\setlength{\parskip}{0pt}
\setlength{\parsep}{1mm}
}
{\end{enumerate}\setlength{\parskip}{11pt}}
%определить среду для компактного перечисления
\newenvironment{citemize}{
\setlength{\parskip}{0pt}
\begin{itemize}
\setlength{\itemsep}{0pt}
\setlength{\parskip}{0pt}
\setlength{\parsep}{1mm}
}
{\end{itemize}\setlength{\parskip}{11pt}}


\setcounter{tocdepth}{2} %Подробность оглавления
%5 это chapter, section, subsection, subsubsection и paragraph
%4 это chapter, section, subsection и subsubsection
%3 это chapter, section, и subsection
%2 это chapter и section
%1 это chapter.
%0 ничего


% С такими полями оно работает по-умолчанию:
%\RequirePackage[left=20mm,right=10mm,top=20mm,bottom=20mm,headsep=0pt]{geometry}
\geometry{right=10mm}
\geometry{left=25mm} 
\geometry{bottom=25mm}
\geometry{top=15mm}
\geometry{headsep=5mm} % чтоб код зачетки не налазил на главы


% чтобы тест не залазил на рамку
\textheight = 710pt % настройка высоты текста, 

% Настройки стиля ГОСТ 7-32
% Для начала определяем, хотим мы или нет, чтобы рисунки и таблицы нумеровались в пределах раздела, или нам нужна сквозная нумерация.
\EqInChapter % формулы будут нумероваться в пределах раздела
\TableInChapter % таблицы будут нумероваться в пределах раздела
\PicInChapter % рисунки будут нумероваться в пределах раздела
%% Если хотите сквозную нумерацию, закомментируйте соответствующую команду

%% настройка листинга
\usepackage{listings}

\usepackage{lstautogobble} % позволяет убирать ведущие символы
\lstdefinestyle{picLstStyle}{
    language=[Sharp]C, % язык программирования
    frame=single, % установка рамки
    xleftmargin=1mm, % отступ слева
    xrightmargin=0mm, % оступ справа
    breaklines=true, % переносить строки
    mathescape=false, % запрет математического режима
    extendedchars=false,
    framesep=2pt,
    keepspaces=true,
    basicstyle=\normalsize,
    aboveskip=1.5em,
}

\lstdefinestyle{LstStyle}{
    language=[Sharp]C, % язык программирования
    frame=none, % без рамки
    xleftmargin=1mm, % отступ слева
    xrightmargin=1mm, % оступ справа
    breaklines=true, % переносить строки
    extendedchars=false,
    keepspaces=true,
    basicstyle=\normalsize,
    keywordstyle=\normalsize
}
% Возможные парамеметры basicstyle , настраиваюшие размер шрифта
%\tiny	
%\scriptsize	
%\footnotesize	
%\small	
%\normalsize	
%\large
%\Large	
%\LARGE	
%\huge	
%\Huge	
\lstset{style=LstStyle}
%%


% настройка отображения источников \nocite означает что они будут добавленны даже если в тексте не использовались ссылки.
% если закомментировать, то отображаться будут только те источники на которых ссылались
% \nocite{*}
%%


% Расскометируйте строчку ниже если хотите добавить фон в весь файл
% Тогда каждой странице исходного файла будет соответсвовать страница файла с фоном
% Размер файла с фонами, в страницах, должен быть >= размеру  исходного файла
% Если будет меньше произойдет фатальная ошибка сборки
% Данный макрос взят отсюда: https://habr.com/ru/articles/163315/
%  \newcommand{\overlaypass}

\usepackage{eso-pic}

\ifdefined\overlaypass
    \newcounter{overlaypage}
    \setcounter{overlaypage}{1}
    \AddToShipoutPicture{%
        \AtPageLowerLeft{% начало координат левый нижний угол
            \put(0,0){% возможность сдвига вставки
                \includegraphics[page=\arabic{overlaypage}]
                    {../Source/background/background.pdf}%
                \stepcounter{overlaypage}
            }
        }
    }
\fi
%%


% установка стиля страниц
\pagestyle{empty}
%%



% По оформлению в многостраничной таблице на последней
% странице заголовок должен иметь отдельное оформление (Окончание таблицы N)
% а также последняя строка на странице не должна закрываться
% если таблица продолжится на следующей.
\subimport{../document/}{preamble-longtable.tex}

% Задание полей для титульника
\subimport{./}{title-page-preamble.tex}